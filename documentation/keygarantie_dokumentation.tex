\documentclass[10pt,a4paper]{article}
\usepackage[utf8]{inputenc}
\usepackage[german]{babel}
\usepackage[T1]{fontenc}
\usepackage{amsmath}
\usepackage{amsfonts}
\usepackage{amssymb}
\usepackage{tabularx}
\usepackage{ltablex}
\usepackage{graphicx}
\graphicspath{{images/}}
\usepackage{subcaption}
\usepackage{cite}
\usepackage[nottoc,numbib]{tocbibind}
\usepackage{hyperref}
\hypersetup{
    colorlinks,
    citecolor=black,
    filecolor=black,
    linkcolor=black,
    urlcolor=black
}
\usepackage{sectsty}

\author{Christopher Besch und Katharina Libner}
\title{Keygarantie Verschlüsselungsverfahren}

\begin{document}

\maketitle
\tableofcontents
\newpage

\section{Abstract}

Keygarantie ist ein symmetrisches 128-bit Stromverschlüsselungsverfahren, das stark an ChaCha angelehnt ist\cite{Bernstein2008}.
Damit basiert es auf XOR, was die Ver- und Entschlüsselung mit dem selben Algorithmus erlaubt.
Für die Generation des mit XOR verwendeten Stromschlüssels wird ein komplexer und praktisch äußerst schwer umkehrbarer Mechanismus aus mehreren mathematischen Methoden verwendet, wodurch selbst bei kompromittiertem Klartext die verwendete Passphrase geheim bleibt.

\section{Einleitung}

Das Kernstück von Keygarantie bildet die XOR-Verschlüsselung, wobei ein Klartext mit einem Stromschlüssel bitweise verschlüsselt wird.
Eine $1$ in dem Stromschlüssel dreht den gegenüberliegenden Bit im Klartext um, eine $0$ verändert ihn nicht.
Wenn nun der so generierte Geheimtext das Verfahren mit dem selben Schlüssel erneut durchläuft, werden die umgedrehten Bits erneut verändert, sodass wieder der Klartext entsteht.
Allerdings ist davon auszugehen, dass Teile des Klartextes kompromittiert werden.
Ist dies der Fall, kann zusammen mit dem öffentlichen Geheimtext auf den verwendeten Stromschlüssel geschlossen werden.
Dadurch wären alle zukünftigen Transmissionen ebenfalls kompromittiert.
Dies wäre fatal.

\medskip
Es muss also davon ausgegangen werden, dass unter Umständen der Stromschlüssel öffentlich bekannt ist.
Daher ist sicherzustellen, dass jeder Teil des Stromschlüssels praktisch nie erneut für die Verschlüsselung mit XOR verwendet wird.
Um dies zu erreichen wird ein 128-bit Schlüssel in einer Matrix über 30 Runden diffundiert.
Dieses Verfahren generiert einen 256-bit Seed.

\medskip
Der Seed wird anschließend in einem Pseudozufallsgenerator eingesetzt, um eine unendliche Menge an Bits für den Stromschlüssels zu generieren.
Praktisch werden allerdings aus Sicherheitsgründen nur 512 Bits pro Seed verwendet, wodurch weniger Information für einen Angriff vorhanden sind.
Diese 512 Bits bilden einen Blockschlüssel.
Daher muss der Klartext in 512-bit Blöcke unterteilt werden, die aufsteigend und eindeutig mit einer Blocknummer nummeriert werden.
Hieraus folgt, dass für jeden einzelnen Block ein anderer Seed für den Pseudozufallsgenerator verwendet werden muss.
Durch diesen Schritt, der mit modernen Methoden praktisch unumkehrbar ist, wird die geforderte Sicherheit garantiert.

\section{Linearer Kongruenzgenerator}

Das Ziel ist es, einen theoretisch unendlich langen Schlüsselstrom zu erzeugen, der von einem Seed abhängt, allerding nicht auf diesen schließen lässt.
Er soll wie zufälliges Rauschen aussehen.
Kein Teil des Schlüsselstroms soll in einem praktischen Bereich erneut vorkommen.
Hierzu wird davon ausgegangen, dass jeder benutzte Seed einzigartig.

\medskip
Der lineare Kongruenzgenerator erzeugt zufällig scheinende Werte $x_i$, die allerdings von ihrem Vorgänger $x_{i-1}$ abhängen.
Daher ergibt sich eine rekursive Funktion:
\begin{align*}
    x_i = \left(\left(x_{i-1} \cdot a\right) + c\right) \text{mod } 2^n
\end{align*}
\begin{itemize}
    \item $n$ sei aus $\mathbb{N}$ und markiert die Bitintervallgrenze in der $x_i$ liegt.
          Für das Intervall gilt also: [0, $n$].
          $n$ ist eine öffentlich bekannte Konstante mit dem Wert $64$.
    \item $c$ sei aus $\mathbb{N}<2$ und ist für dieses Verfahren mit $1$ festgesetzt.
    \item $a$ sei aus $\mathbb{N}<2$ und entspricht $134775813$.
\end{itemize}
Da jeder Wert aus der Menge aller $x_i$ erst durch einen vorherigen Wert entsteht, muss es einen Startwert $x_0$ geben, den Seed.
Der Seed ist abgesehen von den generierten Werten die einzige geheime und einem Angreifer unbekannte Variable.
Da $n = 64$ produziert der Generator Werte zwischen $0$ und $2^{64}$.

\medskip
Dieses Verfahren weist jedoch eine Periode nach spätestens $2^{64}$ WErten auf, nach der sich die Ausgaben wiederholen.
Da jeder Wert durch den Vorgänger determiniert ist, kann jeder pro Periode nur einmal vorkommen.
Dies ist ein Problem, da so ein Angreifer weiß, dass, wenn zum Beispiel die Zahl $143$ verwendet wurde, diese Zahl erst nach Periodenende erneut vorkommen kann.

Um dies zu umgehen wird nur die Hälfte jedes Wertes ausgegeben und für die eigentliche XOR Verschlüsselung eingesetzt.
Intern wird weiterhin der ganze Wert für die Generation des Nächsten verwendet.
Daher wird der Bereich der ausgegebenen Werte auf bis zu $2^{32}$ verkleinert.

So wird die Periode nicht umgangen, weshalb nur eine begrenzte Menge an Werten pro Seed verwendet werden darf.
Das Problem der in einer Periode nicht wiederkehrenden Werte, wurde so allerdings gelöst; es kann nicht genau gesagt werden, wie oft ein bestimmter Wert pro Periode vorkommt.

\section{Matrix}

Das Ziel dieses Schrittes ist es, einen einzigartigen Seed zu generieren.
Dies setzt der Schlüsselstromgenerator voraus.
Zudem sollte dieser Rechenschritt praktisch unumkehrbar ist.
Es wird eine $4 \times 4$ Matrix verwendet, die aus folgenden Teilen besteht:
\begin{itemize}
    \item einer öffentlichen Konstante, sie entspricht der ASCII-Repräsentation von
          \begin{quotation}
              \glqq ILoveYou\grqq{},
          \end{quotation}
    \item dem geheimen Schlüssel,
    \item der öffentlichen Blocknummer und
    \item einer ebenfalls öffentlichen Nonce.
          Die Nonce ist eine \glqq \textbf{N}umber used \textbf{ONCE}\grqq{}, die pro Verschlüsselung neu gewählt werden muss.
          Hierdurch werden Replay-Angriffe praktisch unmöglich, da ein so ein kompromittierter Stromschlüssel nie erneut zur Ver- und Entschlüsselung eingesetzt werden kann.
\end{itemize}

Die Elemente der Matrix sind 16-bit Wörter und sind im Ausgangszustand wie folgt angeordnet:

\begin{center}
    \begin{tabular}{ c c c c }
        const      & const      & const & const \\
        key        & key        & key   & key   \\
        key        & key        & key   & key   \\
        block\_num & block\_num & nonce & nonce \\
    \end{tabular}
\end{center}

Diese Matrix durchläuft 30 Runden an Diffusion, wodurch eine geringfügige Änderung eines der Eingangswörter die gesamte Ausgangsmatrix beeinflusst.

\medskip
Jede Runde besteht aus acht Rechenschritten.
Für alle werden vier Wörter aus der Matrix ausgewählt und dem in \autoref{sec:Achtelrunde} beschriebenen Verfahren übergeben.
Die Ausgaben dieser Berechnung ersetzten die entsprechenden Eingangswerte.

Die Auswahl der vier Wörter ist vordefiniert.
Diese wird von ChaCha übernommen\cite{Bernstein2008}.
Wenn die Wörter zeilenweise nummeriert werden, ergibt sich die folgenden Reihenfolge der Rechenschritte:
\begin{enumerate}
    \item Spalten:
          \begin{itemize}
              \item Achtelrunde(00, 04, 08, 12)
              \item Achtelrunde(01, 05, 09, 13)
              \item Achtelrunde(02, 06, 10, 14)
              \item Achtelrunde(03, 07, 11, 15)
          \end{itemize}
    \item Diagonalen:
          \begin{itemize}
              \item Achtelrunde(00, 05, 10, 15)
              \item Achtelrunde(01, 06, 11, 12)
              \item Achtelrunde(02, 07, 08, 13)
              \item Achtelrunde(03, 04, 09, 14)
          \end{itemize}
\end{enumerate}

\subsection{Achtelrunden}
\label{sec:Achtelrunde}

Jede Achtelrunde nimmt vier 2-bit Wörter ($a$ bis $d$) als Eingabe entgegen und produziert eine Ausgabe von vier 2-bit Wörtern.
Das Ziel dieses Verfahren ist es, alle Eingangswörter derartig miteinander in Beziehung zu setzten, dass eine kleine Änderung eines Eingangswortes alle Ausganswörter ausschalgebend ändert.
Hierzu werden die folgenden Berechnungen durchgeführt:

\begin{align*}
    a & +=       b  \\
    d & \wedge = b  \\
    d & <<<=     6  \\
      &             \\
    c & +=       d  \\
    b & \wedge = c  \\
    c & <<<=     12 \\
      &             \\
    a & +=       b  \\
    d & \wedge = b  \\
    d & <<<=     8  \\
      &             \\
    c & +=       d  \\
    b & \wedge = c  \\
    c & <<<=     7  \\
\end{align*}

Es wird die übliche Schreibweise der C Programmiersprache verwendet.
\cite{Bernstein2008} setzt die selbe ein.
\begin{itemize}
    \item $\wedge$ steht für XOR,
    \item $+$ für Addition modulo $2^{16}$ und
    \item $<<<b$ bitweise Rotation um b-bits nach links.
\end{itemize}

\section{Hashfunktion}

Zur Generation des in der Matrix zu verwendenden Schlüssels wird eine Hashfunktion verwendet.
Sie muss nicht besonders sicher und unumkehrbare sein, da der Schlüsslestromgenerator und die Matrix diese Aufgabe bereits übernehmen.
Der Zweck dieser ist statt dessen die Reduzierung, beziehungsweise Erweiterung, einer Passphrase auf die von der Matrix geforderten 128 Bits.
Daher wird das simple Division-Rest-Verfahren\cite{Silies2021} angewendet.

\begin{itemize}
    \item Sei $s_0$ bis $s_{n-1}$ die ASCII-Repräsentation einer Passphrase der Länge $n$,
    \item $h$ der zu generierende Hashwert und
    \item $p = 2^{128}$ die Begrenzung des Hashwertes, die von der Matrix vorgegeben ist.
\end{itemize}
Es gilt:
\begin{align*}
    h = \left( s_0 \cdot 31^{n-1} + s_1 \cdot 31^{n-2} + ... + s_{n-2} \cdot 31 + s_{n-1}\right)\text{mod } p
\end{align*}

\section{Implementation}

Das genannte Verfahren wird in Java implementiert.
Im Folgenden werden auf Implementationsdetails eingegangen.

Alle Algorithmen wurden funktional implementiert.

Die Eingabedatei wird ausgelesen und in Blöcke unterteilt.
Dabei beträgt die Größe dieser 512 Bits.
Andere Größen sind ebenfalls möglich.
Allerdings muss eine Ballance zwischen
\begin{itemize}
    \item einer hohen kryptographischen Sicherheit bei vielen, kleineren Blöcken und
    \item einer geringen Laufzeit bei wenigeren, größeren Blöcken.
\end{itemize}

Diese Variable Blockgröße ergibt sich aus der theoretisch unbegrenzten Länge der Blockschlüssel.
Zuerst wird die eingegebene Passphrase in einen 128-bit Hashwert konvertiert.

\subsection{Hashfunktion}

Hierzu wird das genannte Division-Rest-Verfahren in der Funktion \textit{division\_hash} rekursiv implementiert.
Es werden immer Substrings der Eingabe gebildet.
Diese unterscheiden sich zu der originalen Eingabe durch die Entfernung des letzten Buchstabens.
Die Funktion wird erneut mit diesem Substring ausgeführt.
Der Rekursionsanker wird getroffen, wenn die Länge der Eingabe $1$ erreicht.

\subsection{Matrix}

Als Repräsentation der 16-bit Wörter werden shorts verwendet, die die Matrix, ein zweidimensionales Array, füllen.
Es werden die folgenden Funktionen genutzt:
\begin{itemize}
    \item \textit{generate\_seed}
    \item \textit{}
    \item \textit{}
\end{itemize}

\subsection{Pseudozufallsgenerator}

Der Generator wird in einer Methode, die den Seed als Parameter entgegennimmt implementiert.
Es wird jeder letzte Wert von $x_i$, aus dem immer der nächste folgt, gespeichert.
Dies ermöglicht es einen 512-bit Blockschlüssel zu erzeugen.

Gehe man davon aus, dass alle Werte $x_i$ den Schlüssel bilden. Sobald also ein Teil oder der ganze Schlüssel bekannt ist, sind ebenfalls die entsprechenden Werte aus der Menge der $x_i$ bekannt. Mit mathematischen Mittel können alle Parameter und damit auch der $seed$ gefunden werden.
Um diesem Sicherheitsproblem zu entgehen, wird dem Schlüssel nur die rechte Bithälfte von jedem Wert $x_i$ zugeordnet.
Der Generator arbeitet jedoch mit dem ganzen Werten von $x_i$ verborgen weiter.
Beispiel:
\begin{align*}
    B_{x_i} = 1001 1111
\end{align*}
\begin{align*}
    B_{x_s} = 1111
\end{align*}
Der Schlüssel $s$ besteht demnach aus allen $x_s$, die aneinandergereiht werden.
\begin{align*}
    s = x_{s_1}, x_{s_2}, x_{s_3}, x_{s_4}, ...
\end{align*}

\subsection{Abscheinden der linken Bithälfte}

Der Wert von $x_s$ entsteht durch eine Art Maske oder Folie, die über $x_i$ gelegt wird.
Mit diesem Instrument lass sich nur noch die rechte Bithälfte von $x_i$ auf $x_s$ abbilden.
Programmiertechnisch werden die Bits von der Maske mit den von $x_i$ verglichen.
Wenn beide an der gleichen Stelle den gleichen Bitwert haben, wird dieser übernommen, ist dies nicht der Fall gilt für $x_s$ an der Stelle 0.
Um nur die rechte Bitseite zu übernehmen, muss also eine gleichlange Bitmaske $m$ erzeugt werden, die von links bis zur Mitte aus Nullen und von der Mitte bis rechts aus Einsen besteht.
Beispiel:
\begin{align*}
    B_{x_i} = 1111 1001SS
\end{align*}
\begin{align*}
    B_{x_i} = 0000 1111
\end{align*}
\begin{align*}
    B_{x_s} = 0000 1001
\end{align*}

\section{Analyse}

\section{Beispiele}

\section{Zusammenfassung}

Keygarantie ist ein experimentelles, nicht empfohlenes, Verschlüsselungsverfahren, das viele Aspekte von ChaCha übernimmt und abwandelt.
Für die praktische Verwendung im Secure Socket Layer ist es nicht empfohlen, da weitreichende Kryptoanalysen noch nicht durchgeführt aber für dieses Anwendungsgebiet notwendig wären.

Ausschließlich für die in RFC 1149 standardisierte Übertragung von IP Datagrams\cite{Waitzman1990} ist dieses Verfahren vertretbar und empfohlen.

\bibliography{database}{}
\bibliographystyle{alpha}

\end{document}
