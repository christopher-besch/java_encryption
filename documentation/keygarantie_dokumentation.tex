\documentclass[10pt,a4paper]{article}
\usepackage[utf8]{inputenc}
\usepackage[german]{babel}
\usepackage[T1]{fontenc}
\usepackage{amsmath}
\usepackage{amsfonts}
\usepackage{amssymb}
\usepackage{tabularx}
\usepackage{ltablex}
\usepackage{graphicx}
\graphicspath{{images/}}
\usepackage{subcaption}
\usepackage{cite}
\usepackage[nottoc,numbib]{tocbibind}
\usepackage{hyperref}
\hypersetup{
    colorlinks,
    citecolor=black,
    filecolor=black,
    linkcolor=black,
    urlcolor=black
}
\usepackage{sectsty}

\author{Christopher Besch und Katharina Libner}
\title{Keygarantie Verschlüsselungsverfahren}

\begin{document}

\maketitle
\tableofcontents
\newpage

\section{Abstract}

Keygarantie ist ein symmetrisches 128-bit Stromverschlüsselungsverfahren, das stark and ChaCha angelehnt ist\cite{Bernstein2008}.
Damit basiert es auf XOR, was die Ver- und Entschlüsselung mit dem selben Algorithmus erlaubt.
Für die Generation des mit XOR verwendeten Stromschlüssels wird ein komplexer Mechanismus aus praktisch äußerst schwer umkehrbaren Methoden verwendet, wodurch selbst bei kompromittiertem Klartext die verwendete Passphrase geheim bleibt.

\section{Einleitung}

Das Kernstück von Keygarantie bildet die XOR-Verschlüsselung, wobei ein Klartext mit einem Stromschlüssel bitweise verschlüsselt wird.
Eine $1$ in dem Stromschlüssel dreht den gegenüberliegenden Bit im Klartext um, eine $0$ verändert ihn nicht.
Wenn nun der so generierte Geheimtext das Verfahren mit dem selben Schlüssel erneut durchläuft, werden die umgedrehten Bits erneut verändert, wodurch der Klartext entsteht.
Hierbei ergibt sich allerdings das Problem, dass davon auszugehen ist, dass Teile des Klartextes kompromittiert werden
Ist dies der Fall, kann zusammen mit dem öffentlichen Geheimtext auf den verwendeten Stromschlüssel geschlossen werden, womit alle zukünftigen Transmissionen ebenfalls kompromittiert sind.

\medskip
Es muss also davon ausgegangen werden, dass unter Umständen der Stromschlüssel öffentlich bekannt ist.
Daher ist sicherzustellen, dass ein bekannter Teil des Stromschlüssels praktisch nie erneut für die Verschlüsselung mit XOR verwendet wird.
Um dies zu erreichen wird ein 128-bit Schlüssel über 30 Runden diffundiert.
Dieses Verfahren generiert einen 256-bit Seed, der praktisch äußerst schwer zu reversieren ist.
Hierdurch wird die geforderte Sicherheit garantiert.

\medskip
Der Seed wird anschließend in einem Pseudozufallsgenerator eingesetzt, um eine unendliche Menge an Bits für den Stromschlüssels zu generieren.
Praktisch werden allerdings aus Sicherheitsgründen nur 512 Bits pro Seed verwendet.
Daher muss der Klartext in 512-bit Blöcke unterteilt werden, die aufsteigend und eindeutig mit einer Blocknummer nummeriert und identifiziert werden.
Hieraus folgt, dass für jeden einzelnen Block ein anderer Seed für den Pseudozufallsgenerator verwendet werden muss.

\section{Die Matrix}

Zur Erzeugung des Seeds wird eine $4 \times 4$ Matrix verwendet, die aus folgenden Teilen besteht:
\begin{itemize}
    \item einer öffentlichen Konstante,
    \item dem geheimen Schlüssel,
    \item der öffentlichen Blocknummer und
    \item einer ebenfalls öffentlichen Nonce.
\end{itemize}
Die Nonce ist eine \glqq \textbf{N}umber used \textbf{ONCE}\grqq{}, die pro Verschlüsselung neu gewählt werden muss.
Hierdurch werden Replay-Angriffe praktisch unmöglich, da ein so ein kompromittierter Stromschlüssel nie erneut eingesetzt werden kann.

Die Elemente der Matrix sind 16-bit Wörter und sind im Ausgangszustand wie folgt angeordnet:

\begin{center}
    \begin{tabular}{ c c c c }
        const      & const      & const & const \\
        key        & key        & key   & key   \\
        key        & key        & key   & key   \\
        block\_num & block\_num & nonce & nonce \\
    \end{tabular}
\end{center}

Diese Matrix durchläuft 30 Runden an Diffusion, wodurch eine geringfügige Änderung eines der Eingangswörter die gesamte Ausgangsmatrix beeinflusst.
Hierdurch wird der Rückschluss auf die Eingangswörter mithilfe des generierten Seeds praktisch unmöglich.

\subsection{Die Runden}

\subsubsection{Achtelrunden}

Jede Achtelrunde nimmt vier 2-bit Wörter ($a$ bis $d$) entgegen und produziert ebenfalls vier 2-bit Wörter.
Das Ziel dieses Verfahren ist es, alle Eingangswörter derartig miteinander in Beziehung zu setzten, dass eine kleine Änderung eines Eingangswortes alle Ausganswörter stark beeinflusst.
Hierzu werden die folgenden Berechnungen durchgeführt:

\begin{align*}
    a & +=       b \\
    d & \wedge = b \\
    d & <<<=     6 \\
      &            \\
    c & +=       c \\
    b & \wedge = b \\
    c & <<<=     6 \\
      &            \\
    a & +=       b \\
    d & \wedge = b \\
    d & <<<=     6 \\
      &            \\
    c & +=       d \\
    b & \wedge = c \\
    c & <<<=     6 \\
\end{align*}

Es wird die übliche Schreibweise der C Programmiersprache verwendet, die ebenfalls in \cite{Bernstein2008} eingesetzt wurde:
$\wedge$ steht für XOR, $+$ für Addition modulo $2^{16}$ und $<<<b$ bitweise Rotation um b-bits nach links.

\section{Pseudozufallsgenerator}

Ziel ist es einen \glqq zufälligen\grqq{} Schlüssel zu erzeugen, der unendlich lang werden kann, um unbeschränkt große Bitanzahlen zu verschlüsseln.

\subsection{Linearer Kongruenzgenerator}

\begin{align*}
    x_i=\left(\left( x_{i-1}\cdot a\right) +c\right) \text{mod } 2^n
\end{align*}

$x_i$ ist ein neuer „Zufallswert“, der dem Schlüssel zugeordnet wird.
$x_{i-1}$ ist dementsprechend der vorherige Wert, der bestimmt wurde.
Da jeder Wert aus der Menge aller $x_i$ erst durch einen vorherigen Wert entsteht, muss es einen Startwert geben: $x_0 = seed $.
$n$ sei aus $ \mathbb{N} $ und markiert die Bitintervallgrenze in der $x_i$ liegt.
Intervall: [0, $n$].
$c$ sei aus  $ \mathbb{N}<2 $
$n$ und dient als Summand.
$a$ sei in $ \mathbb{N}< 2 $
$n$ und dienst als Multiplikator.

\subsection{Enstehung des Schlüssel}

Natürlich erkennt man schnell, dass durch mehrere bekannten Werte aus der Menge der $x_i$ und ein wenig geschickte Mathematik alle Parameter und damit auch der $seed$ gefunden werden können.
Um dem aus dem Weg zu gehen, wird dem Schlüssel nur die rechte Bithälfte von jedem Wert $x_i$ zugeordnet.
Der Generator arbeitet jedoch mit dem ganzen Werten von $x_i$ weiter.
Beispiel:
\begin{align*}
    B_{x_i} = 1001 1111
\end{align*}
\begin{align*}
    B_{x_s} = 1111
\end{align*}
Der Schlüssel $s$ besteht demnach aus allen $x_s$, die aneinandergereiht werden.
\begin{align*}
    s = x_{s_1}, x_{s_2}, x_{s_3}, x_{s_4}, ...
\end{align*}

\subsection{Abscheinden der linken Bithälfte}

Der Wert von $x_s$ entsteht durch eine Art Maske oder Folie, die über $x_i$ gelegt wird.
Diese ist so aufgebaut, dass durch sie nur die rechte Bitseite von $x_i$ durchschimmert und $x_s$ zugeordnet wird.
Im Programm wird mit dem Vergleichsinstrument \flqq and\frqq gearbeitet.
Es werden die Bits von der Maske mit den der von $x_i$ verglichen.
Wenn beide an der gleichen Stelle den gleichen Bitwert haben, wird dieser übernommen, ist dies nicht der Fall gilt für $x_s$ an der Stelle 0.
Um nur die rechte Seite zu übernehmen, muss also eine gleichlange Bitmaske $m$ erzeugt werden, die von links bis zur Mitte aus Nullen und von der Mitte bis rechts aus Einsen besteht.
Beispiel:
\begin{align*}
    B_{x_i} = 1111 1001SS
\end{align*}
\begin{align*}
    B_{x_i} = 0000 1111
\end{align*}
\begin{align*}
    B_{x_s} = 0000 1001
\end{align*}

\subsection{Pseudozufallsgenerator als Klasse}

Der Generator wird als Klasse erzeugt.
Als Eingabe bekommt er die Parameter $a$, $b$, $n$ und
den $seed$.
Der Vorteil an einer Klasse ist, dass jeder letzte Wert von $x_i$ gespeichert wird.
Sodass
man innerhalb der Klasse mit Hilfe einer Methode jeden nächsten Teilschlüssel $k_i$ mit der
Bitanzahl $n$ erzeugen kann.
Und die Möglichkeit besteht den zusammengesetzten Schlüssel $k$
aus den Teilschlüsseln $k_i$ unbeschränkt lang zu wählen.

\section{Analyse}

\section{Beispiele}

\section{Zusammenfassung}

\bibliography{database}{}
\bibliographystyle{alpha}

\end{document}
